\documentclass[10pt]{article}

\usepackage[margin=1in]{geometry}
\usepackage{textcomp}
\usepackage{parskip}
\usepackage[backend=biber, style=alphabetic, sorting=ynt]{biblatex}

\addbibresource{paper.bib}

\begin{document}

\section{Data Formats}

Alpha/neutron discriminators used a variety of different data formats to make predictions. These are collected from the two essential sensors present in the PICO-60 apparatus: piezoelectric microphones (piezos) and cameras.

\subsection{Piezo-Derived}

\subsubsection{Raw Waveform}

The lowest-level data derived from the piezos is the raw waveform $\omega$. $\omega$ consists of a series of samples collected every 400,000\textsuperscript{th} of a second. They are each represented as a 16-bit integer. Only the section of the audio between sample 90,000 (inclusive) and sample 190,000 (exclusive) was used during most experiments, because before this there is background noise, and afterwards there is a clipped signal produced by hydraulics repressurizing the vessel. (These sections contain no information and can be undesirably fit on by neural networks.)

\subsubsection{Fourier Transform}

The raw waveform $\omega$ is converted into the frequency domain by means of a 1-dimensional Discrete Fourier Transform for real input, which is implemented in \texttt{numpy} as \texttt{rfft}. Since DFTs produce sequences of complex numbers, the magnitude of each element is computed. The full-resolution Fourier transform $\beta _{50,000}$ consists of a sequence of complex numbers, half the length of the raw waveform $\omega$.

Beyond this, arbitrary-resolution banded Fourier transforms $\beta _{N<50,000}$ can be computed by integrating the resonant energy over all frequencies within each of a set of bands. They can be thought of as a method of signal downscaling, compressing the signal into a smaller number of data points. Particular band resolutions used include $\beta _{8}$ (which is the input to Acoustic Parameter and initial neural network experiments), $\beta _{5}$, $\beta _{10}$, $\beta _{20}$, and $\beta _{40}$.

\subsection{Camera-Derived}

\subsubsection{Image Window Sequence}

For each event that is captured, the 4 cameras within the PICO-60 apparatus each capture a sequence of 71 images before and after formation of a bubble is detected. These raw images contain a large amount of extraneous information; they encompass the entire vessel. To reduce the input information, the image window sequence $\iota$ includes 50\texttimes50 cropped windows around the position of the bubble. The 71 frames, many of which contain either no bubble or a bubble in later stages of formation, are reduced to 10 immediately around the formation of the bubble. $\iota$ includes 5 frames from before the recording trigger, because the bubble does not cause a trigger until it is already at a significant size.

\subsubsection{3D Position}

The 3-dimensional position $\chi$ of the bubble within the vessel is calculated using triangulation, based on the known positions and angles of the cameras and the position of the bubble within the field of view of each camera. This is used in the banded frequency position correction function $PosCor(\beta _{8}, \chi)$, which corrects $\beta _{8}$ for variations in amplitude which depend on the position of the bubble.

\section{Data Cuts}

Techniques in this study are trained and validated on a number of different data sets. The selection of these sets focused on the fundamental trade-off between quality and quantity of data; by setting a higher standard for the validity of training data, one has less data to train on.

The initial data set $D$ to which these cuts are applied consists of all events recorded during PICO-60 run 2.

\subsection{Basic Quality Cut}

A certain number of cuts are necessarily applied to all data to ensure meaningful results. Otherwise, significant overfitting on biases in the data is likely. This basic quality cut $QualCut(D)$ consists of the following restrictions:

\begin{itemize}
    \item The run was not collected during engineering or testing ($\texttt{run\_type}\neq99$)
    \item Recording was triggered by the camera ($\texttt{trigger\_main}=0$)
    \item Acoustic Parameter is not erroneously large and negative ($log_{10}(\texttt{acoustic\_bubnum})>-100$)
    \item Recorded more than 25 seconds after reaching target pressure ($\texttt{te}>25$)
    \item The bubble position $\chi$ was successfully calculated

    ($[\chi_{X}, \chi_{Y}, \chi_{Z}]\neq[-100, -100, -100]$)
\end{itemize}

\subsection{Bubble Multiplicity Cut}

PICO-60 events which include multiple bubbles are \textit{always} neutron events; alpha particles and WIMP candidates never scatter. Thus, no discriminator is required to handle these events, so they are removed. The bubble multiplicity cut $MultiCut(D)$ consists of the following restrictions:

\begin{itemize}
    \item Either 0 or 1 bubbles are detected based on images from the camera ($\texttt{nbub}<2$)
    \item The number of bubbles approximated using the pressure transducer is close to 1 ($0.7<\texttt{dytranCZ}<1.3$)
\end{itemize}

\subsection{Wall Cut}

Events that occur near the walls of the vessel have acoustic properties which are notably different from events nearer the center of the vessel. It can be desirable for a discriminator to handle these events correctly; however, Acoustic Parameter does not, and neither does the neural network used in the previous PICO-60 paper. Thus, removing wall events allowed for a more meaningful direct comparison between a new neural network and existing techniques.

The complete wall cut $WallCut(D)$ is a composition of the fiducial cut $FidWallCut(D)$ (which makes use of the 3D position $\chi$), the pressure cut $PresWallCut(D)$ (which uses data from the pressure transducer), and the acoustic cut $AcWallCut(D)$ (which uses the banded Fourier transform $\beta _{8}$).

\subsubsection{Fiducial Cut}

The fiducial cut $FidWallCut(D)$ defines an spatial area along the walls of the vessel within which no events are accepted. It makes use of the bubble position $\chi$, the distance from the bubble to the center of the vessel $R$, and the distance from the bubble to the nearest wall, which is defined as $min(\texttt{Dwall}, \texttt{Dwall\_horiz})$. It is defined as follows:

\begin{itemize}
    \item $Z \leq 523$
    \item Any of the following 4 restrictions are true:
    \begin{itemize}
        \item $(min(\texttt{Dwall}, \texttt{Dwall\_horiz}) > 6) \land (0 < \chi_{Z} \leq 400)$
        \item $(min(\texttt{Dwall}, \texttt{Dwall\_horiz}) > 6) \land (\chi_{Z} \leq 0) \land (R \leq 100)$
        \item $(min(\texttt{Dwall}, \texttt{Dwall\_horiz}) > 13) \land (\chi_{Z} \leq 0) \land (R > 100)$
        \item $(min(\texttt{Dwall}, \texttt{Dwall\_horiz}) > 13) \land (\chi_{Z} > 400)$
    \end{itemize}
\end{itemize}

\subsubsection{Pressure Cut}

The pressure cut $PresWallCut(D)$ restricts the pressure detected by the pressure transducer, without any position corrections, to be close to 1 ($0.7<\texttt{dytranC}<1.3$). In combination with the acoustic cut $AcWallCut(D)$, this acts as a backup to the fiducial cut.

\subsubsection{Acoustic Cut}

The acoustic cut $AcWallCut(D)$ is defined using the banded Fourier transform $\beta_{8}$ as follows:

$45 < mean(\beta_{8}[(0, 2), (0, 1)]) < 300$

This takes advantage of differences in the frequency distribution (specifically the first and second frequency bands of the first and third piezos) of wall events and non-wall events.

\section{Performance Analysis}

The performance of neural networks relative to the Acoustic Parameter and the network used in the PICO-60 paper was done using the class-wise standard deviation. This is a variable defined as $C$ below, where $N$ and $A$ are the sets of outputs of the binary discriminator in question (Acoustic Parameter or a neural network), corresponding to the sets of neutrons and alpha particles respectively.

$S=std(N \cup A)$

$C=(std(N \div S) + std(A \div S)) \div 2$

The first step is to calculate the standard deviation S of the union of the two sets. This gives an indication of the scale of the overall distribution. When $N$ and $A$ are divided by $S$, they are normalized so that the standard deviation of their union is equal to 1. While the neural network’s outputs are bounded in the range of 0 to 1 with a sigmoid activation, Acoustic Parameter has a significantly wider range. Normalization of the union prevents this from creating a bias where Acoustic Parameter would produce a higher standard deviation with a similarly proportioned error.

The second step is to calculate the mean of the standard deviations of the normalized sets of neutrons and alpha particles individually. This is an indication of how tightly clustered or widely dispersed the discriminator’s predictions are. Very consistent predictions of $x$ for neutrons and $y$ for alphas (for any $x$ and $y$), with minimal variance off of those specific values, will produce a low class-wise standard deviation.

This performance metric is used rather than a simple accuracy score because it takes in to account not only whether the discriminator is correct or incorrect, but how confident it is. The intent is for a decisive discriminator, which produces a wide separation between the two classes, to be preferred over one that produces a nebulous cloud of outputs with a seemingly arbitrary decision boundary.

\section{Supervised Learning}

Supervised learning was used to directly replace Acoustic Parameter. It takes advantage of labeled sets of neutron and alpha events, which are used as training. Several different network configurations were applied to this task.

\subsection{Convolutional Neural Network for Raw Waveform Analysis}

A convolutional neural network was used for analyzing the raw waveform directly, without any preprocessing whatsoever. This avoids any destruction of information, and should be possible, in theory, for a neural network to handle.

For this task, a very deep 1-dimensional fully convolutional neural network was applied. The architecture was inspired by the M34-res network \cite{verydeepconvnets} for analysis of raw waveforms.

\printbibliography

\end{document}