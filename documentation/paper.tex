\documentclass[12pt]{article}

% Use 1-inch margins even though the font is 12-point
\usepackage[margin=1in]{geometry}
\usepackage{textcomp}

\begin{document}

% Details on the data formats used as inputs for alpha/neutron discriminators
\section{Data Formats}

Alpha/neutron discriminators used a variety of different data formats to make predictions. These are collected from the two essential sensors present in the PICO-60 apparatus: piezoelectric microphones (piezos) and cameras.

\subsection{Piezo-Derived}

\subsubsection{Raw Waveform}

The lowest-level data derived from the piezos is the raw waveform $\omega$. $\omega$ consists of a series of samples collected every 400,000\textsuperscript{th} of a second. They are each represented as a 16-bit integer. Only the section of the audio between sample 90,000 (inclusive) and sample 190,000 (exclusive) was used during most experiments, because before this there is background noise, and afterwards there is a clipped signal produced by hydraulics repressurizing the vessel. (These sections contain no information and can be undesirably fit on by neural networks.)

\subsubsection{Fourier Transform}

The raw waveform $\omega$ is converted into the frequency domain by means of a 1-dimensional Discrete Fourier Transform for real input, which is implemented in \texttt{numpy} as \texttt{rfft}. Since DFTs produce sequences of complex numbers, the magnitude of each element is computed. The full-resolution Fourier transform $\beta _{50,000}$ consists of a sequence of complex numbers, half the length of the raw waveform $\omega$.

Beyond this, arbitrary-resolution banded Fourier transforms $\beta _{N<50,000}$ can be computed by integrating the resonant energy over all frequencies within each of a set of bands. They can be thought of as a method of signal downscaling, compressing the signal into a smaller number of data points. Particular band resolutions used include $\beta _{8}$ (which is the input to Acoustic Parameter and initial neural network experiments), $\beta _{5}$, $\beta _{10}$, $\beta _{20}$, and $\beta _{40}$.

\subsection{Camera-Derived}

\subsubsection{Image Window Sequence}

For each event that is captured, the 4 cameras within the PICO-60 apparatus each capture a sequence of 71 images before and after formation of a bubble is detected. These raw images contain a large amount of extraneous information; they encompass the entire vessel. To reduce the input information, the image window sequence $\iota$ includes 50\texttimes50 cropped windows around the position of the bubble. The 71 frames, many of which contain either no bubble or a bubble in later stages of formation, are reduced to 10 immediately around the formation of the bubble. $\iota$ includes 5 frames from before the recording trigger, because the bubble does not cause a trigger until it is already at a significant size.

\subsubsection{3D Position}

The 3-dimensional position $\chi$ of the bubble within the vessel is calculated using triangulation, based on the known positions and angles of the cameras and the position of the bubble within the field of view of each camera. This is used in the banded frequency position correction function $PosCor(\beta _{8}, \chi)$, which corrects $\beta _{8}$ for variations in amplitude which depend on the position of the bubble.

\section{Data Cuts}

Techniques in this study are trained and validated on a number of different data sets. The selection of these sets focused on the fundamental trade-off between quality and quantity of data; by setting a higher standard for the validity of training data, one has less data to train on.

The initial data set $D$ to which these cuts are applied consists of all events recorded during PICO-60 run 2.

\subsection{Basic Quality Cut}

A certain number of cuts are necessarily applied to all data to ensure meaningful results. Otherwise, significant overfitting on biases in the data is likely. This basic quality cut $QualCut(D)$ consists of the following restrictions:
\begin{itemize}
    \item The run was not collected during engineering or testing ($\texttt{run\_type}\neq99$)
    \item Recording was triggered by the camera ($\texttt{trigger\_main}=0$)
    \item Acoustic Parameter is not erroneously large and negative ($log_{10}(\texttt{acoustic\_bubnum})>-100$)
    \item Recorded more than 25 seconds after reaching target pressure ($\texttt{te}>25$)
    \item The bubble position $\chi$ was successfully calculated ($[X, Y, Z]\neq[-100, -100, -100]$)
\end{itemize}

\subsection{Bubble Multiplicity Cut}

PICO-60 events which include multiple bubbles are \textit{always} neutron events; alpha particles and WIMP candidates never scatter. Thus, no discriminator is required to handle these events, so they are removed. The bubble multiplicity cut $MultiCut(D)$ consists of the following restrictions:
\begin{itemize}
    \item Either 0 or 1 bubbles are detected based on images from the camera ($\texttt{nbub}<2$)
    \item The number of bubbles approximated using the pressure transducer is close to 1 ($0.7<\texttt{dytranCZ}<1.3$)
\end{itemize}

\end{document}