\documentclass[12pt]{article}

\usepackage[margin=1in]{geometry}
\usepackage{textcomp}
\usepackage{parskip}
\usepackage{amsmath}
\usepackage{authblk}
\usepackage[backend=bibtex, style=alphabetic, sorting=ynt]{biblatex}
\usepackage{graphicx}
\usepackage{wrapfig}
\usepackage{float}
\usepackage{caption}
\usepackage{setspace}
\usepackage{fancyhdr}
\usepackage{lastpage}
\usepackage{titlesec}

\pagestyle{fancy}
\renewcommand{\headrulewidth}{0pt}
\renewcommand{\footrulewidth}{0pt}
\fancyhf{}
\lfoot{\scriptsize January 2019}
\cfoot{\scriptsize Brendon Matusch---Improving Particle Classification in WIMP Experiments}
\rfoot{\scriptsize Page~\thepage~of~\pageref{LastPage}}

\graphicspath{{./images/}}

\addbibresource{paper.bib}

\titlespacing{\section}{0pt}{0pt}{0pt}
\titlespacing{\subsection}{0pt}{0pt}{0pt}
\titlespacing{\subsubsection}{0pt}{0pt}{0pt}

\begin{document}

\begin{LARGE}
    \begin{center}
        Improving Particle Classification in WIMP Dark Matter Detection Experiments Using Neural Networks
    \end{center}
\end{LARGE}

\doublespacing

\section{Introduction}

In all experiments for detection of Weakly Interacting Massive Particle (WIMP) dark matter, it is essential to develop a function that can distinguish events caused by WIMP candidates from those caused by background radiation. Manually developing such a classifier is challenging, time-consuming, and necessitates detailed physical modeling.

Machine learning (ML) has the potential to automate this and accelerate experimentation, in addition to detecting patterns that humans cannot. However, impure calibration data hinders training of models, and unusual detector topologies make data challenging to process.

Concretely, I worked with two dark matter experiments: the PICO-60 bubble chamber \cite{pico}, and the DEAP-3600 liquid argon scintillator \cite{deap}. In both experiments, alpha particles comprise a problematic class of background radiation. In PICO-60, WIMP-like neutron calibration events are used as training data; however, alpha particles mixed into this calibration data pose a challenge to effective training. In DEAP-3600, the detector format, containing 255 spherically arranged photomultipliers, makes data processing challenging.

In both experiments, conventional classifiers are typically used. In PICO-60, it is known to be nearly optimal; however, ML was additionally used (for its automation benefits), reaching a mean of 80.2\% accuracy. In DEAP-3600, in a simulated environment, the conventional classifier removes 99.6\% of alpha background radiation, while also (undesirably) removing 91\% of simulated WIMP events.

The objective of my study was to develop novel machine learning algorithms that perform better at this task than previous methods, in the PICO-60 and DEAP-3600 dark matter experiments, and I succeeded at this task. I have lead-authored an academic paper \cite{me} on my PICO-60 research, which the PICO collaboration has reviewed and approved.

\singlespacing
\printbibliography

\end{document}